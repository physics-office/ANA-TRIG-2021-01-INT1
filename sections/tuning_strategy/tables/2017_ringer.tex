

\begin{table}[ht!]\footnotesize
\centering
\caption{Summary of the tuning procedure and decision making strategies employed
to obtain the \rnn for 2017 operation. See text for more
details.}\label{tab:2017_ringer}
\resizebox{\textwidth}{!}{%
\begin{tabular}{p{6cm}p{10cm}}
\hline
\hline
\hline
Criterion & Value \\
\hline
\hline
\multicolumn{2}{c}{Ensemble Composition} \\
\hline
\hline
Model & Fully connected MLP, 1 hidden-layer ($\tanh(.)$) and 1 output ($\tanh(.)$) \\
Phase space bins for Model Tuning &
See Table~\ref{tab:ensemble_regions} \\
%See Tables~\ref{tab:comp_etabins} and \ref{tab:comp_etbins} \\
\hline
\hline
\multicolumn{2}{c}{MLP Training} \\
\hline
\hline
Core Framework & \emph{FastNet} (~\cite{tuningtools}) \\
Dataset and event selection & Simulation (Section~\ref{ssec:dataset})\\
Input Features & Concentric ring sums of energy around the particle axis  \\
Normalization & Absolute of the total ring energies (Section~\ref{top:pp}) \\
Cost-function for tuning & MSE \\
Back-propagation method & RPROP (default parameters, except $\eta^+=1.1$) \\
Targets (Electron/Background) & +1/-1 \\
Batch size & Number of observations in the smaller class \\
Maximum number of training epochs & $\infty$ \\

Over-training evaluation & Multi-stop (see text) using
50 epochs stop criterion \\

Working point reference & Baseline chain efficiencies at \hltcalo \\

Evaluated structures & Number of hidden units ranging from 5 to 20 units \\

Initializations & 100 (Nguyen-Widrow method) \\

Cross-validation method & Stratified k-fold ($k=10$, validation set used for
both testing and early stop computation) \\

Cross-validation subset retrieval method & Split data uniformly into k subsets,
random permutation not applied. Remaining samples are put one by one into the
first subsets \\
ROC extraction method & 1,000 linearly spaced points between model targets \\
\hline
\hline
\multicolumn{2}{c}{Training Evaluation and Operating Model Selection } \\
\hline
\hline

Working Point & ROC point closest to signal efficiency reference and \spmax{}
value \\

Best Initialization Choice & Lowest fake rate (validation set) when operating in
a region of up to $\epsilon=0,2~\%$ of the reference signal efficiency \\

Model Topology Selection & Graphical analysis using box plots \\

Operating Model Choice & Lowest fake rate (using full stats.) when operating in
a region of up to $\epsilon=0,2~\%$ of the reference signal efficiency \\

Operation Extrapolation & Yes, eventual observations outside model phase
space regions use the nearest expert model (in lowest Euclidean
distance in $\et\times\eta$ axis) \\

\hline
\hline
\multicolumn{2}{c}{Decision Making} \\
\hline
\hline

Pile-up Efficiency Correction & Compute straight-line balancing efficiency within
$\avgmu=[0,20,40]$ avg.\@ collisions \\
Dataset and event selection & 2016 $13\;\text{TeV}$ p-p collision data
(GRL: v88), except for reprocessing reference run \\
Decision Making Strategy & Linear fit to the network output without applying the
transfer function w.r.t. $\avgmu$ up to 40$\;$avg.\@ collisions.
When $\avgmu>40$, set $\avgmu = 40$ (2016 upper bound) \\
%Cross-Validation Method & None \\

Phase space bins for Linear Fit & See
Table~\ref{tab:ensemble_regions}\\
%Tables~\ref{tab:comp_etabins} and~\ref{tab:comp_etbins} \\
Target Working Point & Keep HLT signal efficiency of the proposed chain as near
as possible from the respective baseline chain \\
\hline
\hline
\hline
\end{tabular}
}
\end{table}


