


\chapter{The ATLAS Detector}\label{sec:atlas}

\textcolor{red}{
The ATLAS detector~\cite{PERF-2007-01} is a multipurpose particle detector. The inner-detector (ID) system is immersed in a 2T  axial magnetic field and provides charged-particle tracking in the range $|\eta|<2.5$.  Three technologies are employed, with thinner granularity nearer to the interaction point~\cite{PERF-2015-08,CERN-LHCC-97-016,Haywood:331064}: a Pixel Detector (92 million channels), a Silicon Microstrip Detector (SCT, 6.3 million channels) and a Transition Radiation Detector (TRT, 350 thousand 189 channels). The TRT offers electron identification capability within its pseudorapidity coverage ($|\eta|\leq 2$) via the detection of transition-radiation photons generated at the interface between the radiator material and detection straws. Track reconstruction requires (pile-up and the event topology dependence) CPU intensive computations due to the complexity of image-processing algorithms fighting against combinatorial backgrounds when associating hits to tracks~\cite{PERF-2015-08}. In 2016, it was the second most-demanding reconstruction algorithm in the HLT.}

\textcolor{red}{Lead/liquid-argon (LAr) sampling calorimeters provide electromagnetic (EM) energy measurements with fine-grained granularity~\cite{LARG-2009-01,larg_tdr}. A steel/scintillator-tile hadron calorimeter covers the central pseudorapidity range ($|\eta|< 1.7$)~\cite{TCAL-2017-01,tile_tdr}. The endcap and forward regions are instrumented with LAr calorimeters for both the EM and hadronic energy measurements up to $|\eta|< 4.9$. The muon spectrometer surrounds the calorimeters and is based on three large superconducting air-core toroidal magnets with eight coils each.  The muon spectrometer includes a system of precision chambers for tracking and fast detectors for triggering. A two-level trigger system is used to select events~\cite{aad2020performance}. The first-level trigger is implemented in hardware and uses a subset of the detector information to accept events at a rate below 100 kHz on average. An extensive software suite~\cite{ATL-SOFT-PUB-2021-001} is used in the reconstruction and analysis of real and simulated data.
}







\begin{comment}

The ATLAS experiment~\cite{PERF-2007-01} is designed to observe particles
produced in high-energy proton--proton collisions and, hereby not addressed,
heavy-ion LHC collisions. Its calorimeter system comprises both 
electromagnetic (ECAL) and hadronic (HCAL)
sections (Section~\ref{ssec:calo}), which \textcolor{red}{enveloped} an inner \textcolor{red}{tracking system (Section~\ref{sec:track})}. The two-level triggering system reduces data-taking rate \textcolor{red}{from 40 MHz} to approximately \textcolor{red}{1 kHz. Here}, we focus on electron-based channels, which are
found in many interesting physics phenomena, \textcolor{red}{such as the} the decays of the Higgs
boson~\cite{HIGG-2012-27,HIGG-2016-33}, for instance.


The trigger system is discussed in Section~\ref{sec:atlas_trigger}.
The standard variables for electron identification are discussed in
Section~\ref{ssec:std_variables}. The electron triggers are described, in details, in
Section~\ref{ssec:egamma_trigger}, followed by their nomenclature in
Section~\ref{ssec:menu}.  




\section{Calorimeter System}\label{ssec:calo}

The calorimeter system covers the pseudorapidity
range \(|\eta| < 4.9\)~\cite{PERF-2007-01}. Within the region \(|\eta|< 3.2\),
electromagnetic calorimetry is provided by barrel and endcap high-granularity
lead/Liquid-Argon (LAr) calorimeters, with an additional thin LAr presampler
(\ps) covering \(|\eta| < 1.8\), in order to correct for energy losses in
material upstream of the calorimeters~\cite{LARG-2009-01,larg_tdr}. Hadronic
calorimetry is provided by \textcolor{red}{a} the steel/scintillating-tile calorimeter
(\tilecal~\cite{TCAL-2017-01,tile_tdr}), \textcolor{red}{constituted} of three barrel structures
within \(|\eta| < 1.7\), and two copper/LAr hadronic endcap calorimeters
(\hec)~\cite{cal_tdr}.  The solid angle coverage is completed with forward
copper/LAr and tungsten/LAr calorimeter modules optimised for electromagnetic
(EM) and hadronic (HAD) measurements, respectively. The specified calorimeters
provide full azimuthal ($\phi$) coverage with a total of about 190,000 \textcolor{red}{readout cells. Transition regions between calorimeters are used to locate detector services and induce a sharing of showers between calorimeters that degrades energy measurements in those regions. In particular,} the transition from the barrel to the end-cap within
$1.37<|\eta|<1.52$ is complemented by scintillator stab ($1.0<|\eta|<1.6$) to
\textcolor{red}{estimate signal} losses~\cite{cal_tdr}.

The electromagnetic and hadronic systems comprise three sampling layers each along longitudinal (depth) segmentation ~\cite{PERF-2007-01}\footnote{The two
central layers in HCAL end-cap are summed to result in a single measurement.}.
Each sampling layer has its own lateral ($\eta\times\phi$) granularity, for detailed information). Besides granularity, the amount of material in front of the detector in each sampling \textcolor{red}{layer also} vary with \abseta, leading to variable expected lateral and longitudinal profiles. \textcolor{red}{Additionally}, the expected
profile is also dependent on the physics object total energy. \textcolor{red}{In the EM calorimeter}, most of \textcolor{red}{the EM shower} energy is collected in the second layer, while the third layer provides measurements
of energy deposited in the shower tails. %as shown in Figures~\ref{fig:cal_em_x0} and~\ref{fig:cal_had_lambda}.

The hadronic calorimeters, which surround the EM detectors,
provide additional discrimination power through further energy measurements of
possible EM shower tails, as well as rejection of events with activity of
hadronic origin with three sampling layers. \textcolor{red}{It is worth noting that QCD jets also have a sizeable fraction of their energy deposited in the EM calorimeter from $\pi^{0}\rightarrow\gamma\gamma$
and other prompt hadronic decays to photons and also bremsstrahlung $\gamma$ radiated by quarks.}
Although calorimeters are designed 
to have uniform response for full detector acceptance, there are residual fluctuations 
dependent on the shower development position~\cite{Wigmans2017}.

\section{Inner Detector}\label{sec:track}

The inner detector, \textcolor{red}{used to reconstruct charged-particle tracks}, is immersed in a
\textcolor{red}{\SI{2}{\tesla} magnetic} field \textcolor{red}{aligned on the beam axis and produced} by a thin superconducting solenoid
and covers a pseudorapidity range $|\eta| \lesssim 2.5$~\cite{PERF-2007-01}.
Three technologies are employed, with thinner granularity nearer to the
interaction point~\cite{PERF-2015-08,inner_tdr1,inner_tdr2}: \textcolor{red}{a} Pixel Detector (92
million channels), \textcolor{red}{a} Silicon Microstrip Detector (SCT, 6.3 million channels) and
\textcolor{red}{a} Transition Radiation Detector (TRT, 350 thousand channels). The TRT offers
electron identification capability within its pseudorapidity coverage ($|\eta|
\lesssim 2$) via the detection of transition-radiation photons \textcolor{red}{generated at the interface between the radiator material and detection straws.} Track reconstruction requires (pile-up \textcolor{red}{and the event topology} dependence) CPU intensive
computations due to the complexity of image-processing algorithms \textcolor{red}{fighting against combinatorial backgrounds when associating hits to tracks ~\cite{PERF-2015-08}. In} 2016, it was the second
most-demanding reconstruction algorithm in the \hlt{}.


\end{comment}