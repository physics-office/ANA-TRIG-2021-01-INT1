\chapter{Conclusion and Perspectives}\label{sec:conclusion}
%-------------------------------------------------------------------------------






% Comparison with standard strategies
The Neural\rnn{} brings an innovative design for electron selection based on
calorimetry information. By alleviating the implicit requirement of conceiving
variables to exploit individual properties of the particle interaction process, \textcolor{red}{it introduces} 
more general variables exploiting the \textcolor{red}{process natural symmetries.} With the
ambition to comply with specific trigger system demands, the feature extraction
of concentric ring energy sums provides an efficient description, which is quickly derived
through simple operations. Neural network models are employed for exploiting
nonlinear correlations between rings. The algorithm shows
high agreement with the offline and \textcolor{red}{last trigger selection (conventional) methods,} both based
on a maximum likelihood approach. It is one of the motivations whose culminated
in the employment of an ensemble of specialist models per regions of
pseudorapidity and transverse energy. Likewise, the Neural\rnn{} algorithm preserves
better the signal efficiency with respect to pile-up with respect to the
previously cut-based strategy it replaced.



In the second half of 2017, the Neural\rnn{} algorithm replaced \textcolor{red}{such a cut-based} strategy
in the \fastcalo{} decision step and became the baseline algorithm
for triggering events containing electrons above \SI{15}{\GeV}, as part \textcolor{red}{of the ATLAS online system CPU} reduction campaign. The \fastcalo{} selection is crucial for CPU demands, as a more
efficient algorithm allows to avoid heavier computations in the
sequential steps.  It was the first time a neural network method was employed as
a baseline selection algorithm for event selection in the ATLAS trigger system.
During 2017 operation, based on neural networks trained with simulated data,
reduction factors of 13 in the \fastcalo{} and of 2 in
the \hlt{} fake rates were achieved with respect to the offline likelihood, while
keeping high electron efficiency when the Neural\rnn{} was set to operate in a
configuration similar to lowest energy-threshold unprescaled single electron
trigger. It resulted in a reduction of \SI{60}{\%} in the CPU demands \textcolor{red}{for this particular} trigger, when it was set to operate alone. On the trigger system level,
estimated results on reprocessed data show a reduction of \SI{8}{\%} in the CPU
demands of \egamma{} triggers. For 2018, the Neural\rnn{} was trained with collision
data resulting in an estimated additional reduction factor of 1.35 for the
mentioned trigger.

% Low-level information perspectives
\textcolor{red}{It should be emphasized that}, 
%We emphasize that, 
despite being based on a machine learning method, the Neural\rnn{}
contrasts \textcolor{red}{with standard} data-oriented developments for the importance of the
application context and domain knowledge in its design. The goal of the ATLAS
experiment is to comply with its physics programme, i.e.\@ to provide insights on
complex physics processes. In order to achieve such physics understanding, it is
more interesting to design a sufficiently efficient selector while preserving
interpretability of its behavior than reaching ultimate selection efficiency.
Particularly, fully data-oriented approaches may achieve higher efficiencies but
mostly compromise the understanding of the model behavior. On the other edge,
a cut-based strategy based on discriminating variables to exploit individual
properties of the \textcolor{red}{process allows a lower efficiency to draw conclusive} evaluations due to the \textcolor{red}{available} statistics. The optimum strategy is to provide as
much understanding as possible with enough efficiency, a point that might lie
within the \textcolor{red}{two individual approaches. We} believe that a good strategy to achieve it is by
finding an optimal level of \textcolor{red}{constraints} based on domain knowledge to the model
possibilities. The main one in the Neural\rnn{} is the ring sums themselves, which
disallow the networks to exploit more complex patterns. Despite having hints
that more discriminating information might exist, a representation learning
approach would result in patterns difficult to interpret.

% Machine learning, extrapolation
Another claim for designing models accounting for the application context
is the extrapolation to other operating \textcolor{red}{conditions, which is an important challenge in} 
the development \textcolor{red}{ of the Neural\rnn. In the particular case of }
%of the Neural\rnn{}. There are two usual cases in HEP experiments.
%Many physics processes decay into physics objects resulting in atypical data,
%e.g.\@ high kinematic regimes. In the particular case of 
trigger developments, the ever-increasing luminosity makes \textcolor{red}{the experiments operate} in harsher and
unprecedented conditions. Since there is few or no data available, it requires
domain knowledge to allow predicting and, specially, customizing the model to
operate under these conditions. Relying on simulation also requires domain
knowledge for defining the generation process which may be \textcolor{red}{imperfect, potentially compromising the model} extrapolation.

% Quadrant and agreement analysis
The Neural\rnn{} was also studied through the offline likelihood perspective,
reference for physics analysis, to bring additional insights on its behavior. By
taking advantage of the low dimensionality, low correlation and high
interpretability power of the variables employed in the likelihood, we derived
the profiles of the disjoint decision cases of the 2017 duplicate trigger pair.
The choice for the duplicated pair considered the typical scenario employed for
analysis with the setup that would maximize the disagreement between both
configurations: with and without the Neural\rnn{}. The proposed quadrant analysis
allowed to observe high agreement between both triggers with slight shifts
towards the signal region in the disagreement profiles of some calorimetry-based
variables. By checking the agreement between the trigger pair for the derivation
of the likelihood pdfs, the alteration using the full 2017 data was much lower
\textcolor{red}{than} estimated statistical fluctuations. The residuals were found to be bounded
by \SI{0.2}{$\sigma$} for all variables. When considering only the
disagreement profiles, homogeneity hypothesis at a \SI{5}{\%} level was not
rejected for the calorimetry variables, except for \reta{} and \rhad{} despite
using the most populated regions. For these variables, a shift towards signal
region was also observed. Hence, the Neural\rnn{} had negligible impact in the
sensitive variables employed for physics analysis and in the offline selection
operation.

Finally, \textcolor{red}{the presented results show} that it is possible \textcolor{red}{to tighten the requirements} for online operation while still resulting in high agreement with the offline
selection. More generally, it demonstrates that designing a solution \textcolor{red}{from scratch} using the application specificities can provide new effective
strategies. We hope that it can serve as motivation for other developments in
HEP, particularly when considering the trigger side.

\section{Perspective}







% Maximizing contributions to the trigger system: low et and photons
Preliminary evaluations have shown the potential of the Neural\rnn{} algorithm for
triggering on electrons with $\et<\SI{15}{\GeV}$. This is particularly motivating given that triggers in this kinematic region are very
CPU demanding. Additionally, they have high output rate and are usually
prescaled, thus Neural\rnn{} might also allow to collect additional data for those
triggers.


%Photons are other physics objects that may be triggered using the
%\rnn{} approach. Triggers combining photons to other physics objects are quite
%CPU demanding, thus motivating studies to evaluate whether additional Neural\rnn{}
%developments will contribute to increase performance.

Photons are other physics objects that may be triggered using the Ringer approach. Triggers combining photons to other physics objects are quite CPU demanding, thus motivating studies to evaluate whether additional Ringer developments will contribute to increased performance.
When considering pure
photon triggers, the ring sums with linear classifiers can
be investigated in order to keep the analysis strategies employed in channels as
\textcolor{blue}{$\text{H}\rightarrow Z\gamma$}. Another possibility for those triggers is to
employ the Neural\rnn{} for increasing the trigger (signal) efficiency. A possible
setup is to duplicate the photon triggers so that the Neural\rnn{} recovers
interesting events.  In this hybrid menu configuration, the standard trigger
would be available for those analyses so that they may benefit from high
interpretability and customization, whereas the Neural\rnn{} trigger might be employed
for those analyses demanding more statistics. It is expected that this
configuration will demand minor additional resources but may allow higher
statistical significance of the results of physics analysis.

%However, combined photon triggers must be considered carefully. Specifically, if we the Neural\rnn{} is set to
%operate only for very demanding combined photon triggers, the decision
%correlation on fakes for triggers with and without the Neural\rnn{} will play an
%important role in the CPU impact because of decision caching in case
%these triggers have high concurrence rate. Both low-\et{} and photon studies are
%comprised in two ATLAS authorship qualification tasks, which are ongoing.

%\footnote{A list for
%	the most-demanding triggers during Run~2 is available on
%	Table~\ref{tab:top_cpu_photons} (Section~\ref{ssec:menu_cpu}).}

% Linear ringer and hybrid menu
%When considering pure
%photon triggers, the ring sums with linear classifiers can
%be investigated in order to keep the analysis strategies employed in channels as
%$\text{H}\rightarrow\gamma\gamma$. Another possibility for those triggers is to
%employ the Neural\rnn{} for increasing the trigger (signal) efficiency. A possible
%setup is to duplicate the photon triggers so that the Neural\rnn{} recovers
%interesting events.  In this hybrid menu configuration, the standard trigger
%would be available for those \textcolor{red}{analyses} so that they may benefit from high
%interpretability and customization, whereas the Neural\rnn{} trigger might be employed
%for those analyses demanding more statistics. It is expected that this
%configuration will demand minor additional resources but may allow higher
%statistical significance of the results of physics analysis.


% Maximizing CPU/signal efficiency: L0 phase II, L1 phase I
%Along the same line (improving the trigger system performance), a natural path
%is to deploy a strategy similar to the Neural\rnn{} in the hardware based selection.
%Particularly, we consider deployment in the L0 of Phase II upgrade programme for
%which access to full granularity will be available. Additionally, we are
%investigating the possibilities for the Neural\rnn{} employment with the super-cells.
%It is worth mentioning that the operation of Neural\rnn{} at the earlier stages can be
%useful not only for reducing \hlt{} farm demands, but eventually also for
%improving the overall efficiency as the \licalo{} may act as an inefficiency
%source for some triggers~\cite{the_preprint}.

% No-had
Another interesting possibility is to employ the Neural\rnn{} accessing only
electromagnetic information for triggers resulting in high readout rate from
hadronic calorimeter cells. The particular setup may even consider employing a
two step decision on the \fastcalo{}, a preliminary one based solely on
electromagnetic information followed by the full Neural\rnn{} decision.




% Boosted configurations
Physics beyond the Standard Model is getting more \textcolor{red}{attention at the LHC  and is expected} to be part of the main \textcolor{red}{focus}in the data-taking phases to
come. We aim at improving the Neural\rnn{} algorithm to comprise boosted
configurations, particularly limiting the effect of other contributions at
the edge of the feature extraction window, which can result from these physics
processes. Additionally, the Neural\rnn{} is being evaluated as an alternative
strategy for a long-lived particle analysis
($\text{a}\rightarrow\text{H}(\gamma\gamma)Z(l^+l^-)$).

% Improve efficiency: Improve ring description (asymetry), (fusion with shower
% shapes), deep learning
\textcolor{red}{Nevertheless, we expect}
%We expect 
that the Neural\rnn{} efficiency \textcolor{red}{can still be improved. Indeed, the feature} extraction
algorithm has many limitations as being initially proposed for online
operation, but a more complex algorithm accounting for the cell sizes can
provide more precise information, which, eventually, can be helpful for the
selection task. Shower asymmetries may be captured if extracting ring segments,
i.e.\@ quarter rings delimited by $\eta\times\phi$ axis. Complementary
discriminant information may be obtained by fusing the ring sums with the shower shape variables,
which is under investigation. Deep learning models can be exploited instead of
single hidden-layer MLPs, in order to achieve better suited decision boundaries,
specially considering models designed to exploit the sequential structure
presented by the rings.


% Calibration
The shower description by ring sums goes beyond capturing discriminant
information and may be used for calibration of the energy of electrons and
photons. While preliminary results with ring sums are motivating, we expect the aforementioned
improvements in the ring description to be exceptionally important for
calibration.

% Offline developments
The Neural\rnn{} software has been extended to the offline framework. Currently,
the ring description is available for all electrons with \textcolor{red}{$\et>\SI{14}{\GeV}$.
This is an on-going} activity.

