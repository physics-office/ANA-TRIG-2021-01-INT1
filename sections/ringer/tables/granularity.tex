
\begin{table}[ht!]
\centering
\caption{Nomenclature defining the \textcolor{red}{\rnn{}} algorithm layers and sections, as well
  as the respective parameters employed and calorimeter sampling
  layers from which the cells are extracted. The \textcolor{red}{\rnn{}} algorithm
  parameters are dependent on the ringer layer $l$ and independent on \eta{} and
  \et{} during Run~2. The parameters are the ring size in \eta{}
  ($h_{\etaa,l}$), $\phi$ ($h_{\phii,l}$) and the number of rings to be computed
  in each layer ($\text{N}_l$). 
  %The values for $h_{\etaa,l}$ and $h_{\phii,l}$
  %are approximated, the exact ones can be obtained for $\eta=0$ in
  %Table.
%To refer to a ring, it will be used notation XYYY, where X is the ring index in
%the YYY layer.
}%
\label{tab:ring_alg_parameters}
\resizebox{.8\textwidth}{!}{%
\begin{tabular}{lc|ccc|ccc}
\hline
\hline
\multicolumn{2}{c|}{Ringer} & \multicolumn{3}{c|}{Calorimeter Sampling} & 
\multicolumn{3}{c}{Parameters} \\
\hline
Section & Layer ($l$) & Barrel & \itc & End-cap & $h_{\etaa,l}$ & $h_{\phii,l}$ & $N_l$ \\
\hline
\hline
\multirow{4}*{EM} & \ps &  \presamplerb & & \presamplere & 0.025 & 0.1 & 8 \\
\cline{2-5}
& \emi & \emb{1} &  & \emec{1} & 0.003 & 0.1 & 64  \\
\cline{3-5}
& \emii & \emb{2} &  & \emec{2} & 0.025 & 0.025 & 8 \\
\cline{3-5}
& \emiii & \emb{3} &  & \emec{3} & 0.050 & 0.025 & 8 \\
\cline{1-5}
\multirow{6}*{HAD} & \multirow{2}*{\hadi} & \tilebar{0} &
\multirow{2}*{\tilegap{3}} & \multirow{2}*{\hec{0}} & \multirow{2}*{0.1} & \multirow{2}*{0.1} & \multirow{2}*{4} \\
&                     & \tileext{0} &                               &                           \\
\cline{3-5}
& \multirow{2}*{\hadii} & \tilebar{1} & \multirow{2}*{\tilegap{1}} & \hec{1}       & \multirow{2}*{0.1} & \multirow{2}*{0.1} & \multirow{2}*{4} \\
&                   & \tileext{1} &              & \hec{2}  \\
\cline{3-5}
& \multirow{2}*{\hadiii} & \tilebar{2} & \multirow{2}*{\tilegap{2}} & \multirow{2}*{\hec{3}} & \multirow{2}*{0.2} & \multirow{2}*{0.1} & \multirow{2}*{4} \\
&                     & \tileext{2} &                &             \\
\hline
\hline
\end{tabular}
}
\end{table}


