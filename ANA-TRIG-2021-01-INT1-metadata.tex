%-------------------------------------------------------------------------------
% This file contains the title, author and abstract.
% It also contains all relevant document numbers used for an ATLAS note.
%-------------------------------------------------------------------------------

% Title
%\AtlasTitle{\rnn{}: An Ensemble of Neural Networks Fed from Calorimeter
%Ring Sums for Triggering on Electrons}

\AtlasTitle{The ATLAS \rnn{} Algorithm: An Ensemble of Neural Networks Fed from Calorimeter Ring Sums for Triggering on Electrons}

% Draft version:
% Should be 1.0 for the first circulation, and 2.0 for the second circulation.
% If given, adds draft version on front page, a 'DRAFT' box on top of each other page, 
% and line numbers.
% Comment or remove in final version.
\AtlasVersion{0.1}

% Abstract - % directly after { is important for correct indentation
\AtlasAbstract{%
\textcolor{red}{The implementation of an ensemble of neural networks (\rnn{}) in the electron selection at the High Level Trigger (HLT) of the ATLAS experiment is presented. It is dedicated to early removal of fake electrons in its first, fast, selection stage. In the second} half of 2017, the
\rnn{} was set to operate for the selection of electrons above 15 GeV, as part of an effort to reduce CPU usage to comply with more
stringent data taking conditions. Inspired by the ensemble of likelihood models
currently operating in the \textcolor{red}{electron offline selection algorithms}, the \rnn{} ensemble
\textcolor{red}{is built from} Multi-Layer Perceptron (MLP) models tuned on pseudo-rapidity and
transverse energy bins, in order to minimize trigger performance impacts from both
detector response and shower development \textcolor{red}{energy dependencies}. The MLPs are fed from calorimetry information
formatted into energy sums of concentric rings built around the particle
axis and normalized by its total transverse energy. This note describes the
algorithm, its operation efficiency and the results from the analyses done for its characterization. Using this algorithm results in a 60 \% decrease in CPU demands for the lowest-threshold unprescaled
single-electron trigger, while maintaining electron efficiency nearly unchanged.
The \textcolor{red}{analysis} results show an overall CPU saving of 8 \%, when considering the
demands of all electron and photon triggers. \textcolor{red}{This increase of performance  results from a higher jet background rejection with respect to previous algorithms}.
}





% Author - this does not work with revtex (add it after \begin{document})
\author{The ATLAS Collaboration}
% Authors and list of contributors to the analysis
% \AtlasAuthorContributor also adds the name to the author list
% Include package latex/atlascontribute to use this
% Use authblk package if there are multiple authors, which is included by latex/atlascontribute
% \usepackage{authblk}
% % Use the following 3 lines to have all institutes on one line
% \makeatletter
% \renewcommand\AB@affilsepx{, \protect\Affilfont}
% \makeatother
% \renewcommand\Authands{, } % avoid ``. and'' for last author
% \renewcommand\Affilfont{\itshape\small} % affiliation formatting
% \AtlasAuthorContributor{W. S. Freund}{a,c}{Note editing, model development,
% analysis development and supervision}
% \AtlasAuthorContributor{J. V. F. Pinto}{a}{Note editing, model development,
% ATHENA framework developments, analysis development}
% \AtlasAuthorContributor{M. V. Araujo}{a}{Note editing, low \et developments}
% \AtlasAuthorContributor{J. M. Seixas}{a}{Model development, PhD supervisor of W.
% S. Freund and J. V. F. Pinto, note editing}
% \AtlasAuthorContributor{D. O. Damazio}{b}{Contact person of trigger framework,
% ATHENA framework developments}
% \affil[a]{Signal Processing Laboratory (LPS), COPPE/POLI-UFRJ}
% \affil[b]{Brookhaven National Laboratory (BNL)}
% \affil[c]{\emph{Laboratoire de Physique Nucléaire et de Hautes Energies} (LPNHE), Sorbonne University}




% ATLAS reference code, to help ATLAS members to locate the paper
\AtlasRefCode{ANA-TRIG-2021-01}

% ATLAS note number. Can be an COM, INT, PUB or CONF note
\AtlasNote{ANA-TRIG-2021-01-INT1}
