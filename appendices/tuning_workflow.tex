\section{Tuning Work-flow}\label{sec:workflow}


%An overview of the tuning work-flow is available in
%\figurename~\ref{fig:tuning_workflow}.
% TODO First, events are selected to ROOT files either using the offline or
% online frameworks.
%The event selection can be performed by was performed by the online
%(TrigEgammaAnalysis) framework, except for the developments of the low \et
%chains (Section~\ref{sec:low_et}), which used the offline (TagAndProbeFrame)
%framework. The datasets used and event selection strategy is available in
%Table~\ref{tab:event_selection}.
The development is carried out outside the ATHENA framework, on two developed
frameworks. The TuningTools framework allows to approximate inference by
training neural-networks either on custom \emph{C++} code (FastNet) or using the
Keras framework.  The models are selected and then used
under the prometheus framework to fine-tune the working points by minimizing their
differences with the final HLT efficiencies.  Confirmation is carried out
through emulation of the chains on the ATHENA framework and, if they match the
benchmark efficiencies of the baseline chains, the results are reported to the
collaboration. The standard evaluation can be applied next: reprocessing
validation, commissioning stage and, finally, set the chains for operation.

%\begin{figure}[h!t]
%\centering
%\includegraphics[width=\textwidth]{tuning_workflow}
%\caption{\label{fig:tuning_workflow}
%Illustration of the tuning work-flow and steps involved. The circular arrows
%around a step is used to flag that it is manually evaluated recursively until
%satisfactory results are obtained.
%}
%\end{figure}
