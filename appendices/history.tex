\section{Some Words on the ATLAS \rnn History}\label{sec:history}

The inspiration for representing of calorimetry information through concentric
ring sums is based on the SPACAL prototype
calorimeter~\cite{1992_spacal_rings}\footnote{Technology that has been
established as a standard calorimetry technique, which has been employed by a
number of experiments (for instance, H1 and Chorus).}.
The prototype had 155 towers defined by a central unit surrounded through 7
concentric rings. This structure was exploited to provide complementary
measurements through analogue sums of the tower signals composing each ring
using half of their anode signals. Having this information allowed to reduce
efforts in equalization of arrival times of the towers, where the 5 outermost
rings towers could only be studied using their summed signals. The outer ring
sums were employed in sparse data readout, where signals below \SI{5}{MeV} were
cutoff. The rings were evaluated for studies of regression of the position of
the particle interaction and electron-pion identification.

Two authors of the SPACAL prototype joined the ATLAS collaboration in early
1990's and proposed the ring sums together with neural-networks for electron-jet
identification in the second-level trigger system based only on
calorimetry~\cite{1995_seixas_ringer}. This proposal counted with contributions
of many
authors~\cite{1998_anjos_dantas,2003_anjos,2010_torres,2010_simas,2012_ciodaro}
who also worked in its development, implementation and maintenance before and
during Run 1.

It was found another independent work from the same time scope of the SPACAL
prototype that proposed the use of rings with neural-networks for online
electron identification, but for the D$\emptyset$ experiment. The author
referred to the ring sum as radial bins about the shower peak, but a later
neural-network tutorial~\cite{denby1990neural} cited this work while mentioning
to the structure as rings.
