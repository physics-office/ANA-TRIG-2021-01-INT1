\section{Alessandro Tricoli's Comments}%
\label{sec:alessandro_comments}

These comments concern a previous unpublished note with the \rnn{} results in
2016. We place them here as we are addressing them on this note.

\begin{itemize}
\item it is not clear in the text why this new technique is applied on
the fast HLT only and why it may give better performance than the current
technique, given that the metric is given wrt the current offline technique that
does not implement the Ringer. In other words, if the Ringer is indeed superior
to the sliding window it should be implemented offline too and in the precision
HLT step too, why this is not discussed in the note? I am not say it should be
implemented in these two cases, but it should be discussed why the Ringer has
been studied only in the fast HLT step. I guess it is down as a use case and
further studies will follow. Please clarify the text.

\item The understanding of the note relies on the knowledge of the calorimeter
sampling. This information is not given in much detail on the note. I believe a
section should be added with a detailed the description of the calorimeter
technology, segmentation both longitudinally and laterally. Without this
detailed information it is difficult for a non-ATLAS reader to appreciate the
Ringer method.

\item The speed of this new method wrt previous one is not discussed in detail in
the note, but it is critical to judge to usefulness and applicability of this
algorithm at trigger level. More space should be dedicated to the speed and
memory consumption of the Ringer wrt sliding window, you may even add a plot or
two to back this up.

\item The discriminating variables are never shown. I think it is important to show
these distributions, especially to compare their discriminating power wrt shower
shapes. Why a NN is necessary to discriminate over them? wouldn't a simple
cut-based approach  be sufficient? since the Ringer performances are compared to
those of the fast calorimeter selection which implements a cut-based approach,
it would be appropriate to compare the capabilities of the Ringer with a cut
based approach as well to see how much of the improvement comes form the
optimisation of the NN discriminant and how much in the different conception of
the shower shape/ring. This will be important in order to justify any
improvement wrt the current technique.

\item Related to the previous comment, what is the dependence of the efficiencies
and fake rejection on the training of the sample based on Z->ee events? I would
have found more appropriate to train the events on Z->ee events then test the
efficiencies on other signal processes, e.g. H->WW or H->ZZ, H->gamma gamma or
Z'->ee, or other benchmarks.

\item Related to the above questions, how dependent are the efficiency and fake rate
results on the pileup range value used in the training sample? in other words,
if you choose other pileup values how well does the Ringer algorithm compare
with the current clustering algorithm in terms of efficiency and rejection?

\item Photons  are never discussed here, but if you change the clustering and
selection of the fast calorimeter step you need to change this for the photons
too at trigger level. I think you need to spend a few words discussing this and
possibly say that this study focuses on electron performance.

\item The text should improve as in a few parts is unclear. The current text is a
copy of the proceedings for  the ATAC conference, but for an ATLAS note we need
to expand the text and make it more inline with ATLAS standards. Line-to-line
suggestions will follow. I also suggest a native speaker to go through the
English and in some cases it seems can be improved.
\end{itemize}

